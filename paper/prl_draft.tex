\documentclass[aps,prl,reprint,superscriptaddress]{revtex4-2}
\usepackage{graphicx}
\usepackage{amsmath,amssymb}
\usepackage{hyperref}

\begin{document}

\title{Consistency-Driven Causal Graphs Exhibit Emergent Geometry and Discrete Transport Resonances}

\author{(Authors omitted for review)}
\affiliation{(Affiliations omitted for review)}

\begin{abstract}
We study a minimal ``computational universe'' in which a directed causal graph is annealed under a local consistency energy that penalizes incompatible motifs.
Starting from random layered directed acyclic graphs, the annealing self-organizes a manifold-like causal structure with predominantly polynomial light-cone growth and an effective dimension $d_{\mathrm{eff}}\sim 2$--$3$.
We then probe the resulting disordered causal medium by (i) measuring horizon-limited light-cone volumes and (ii) evolving simple local field rules after injecting compact defects.
Near a narrow connectivity window ($p_{\mathrm f}\approx 0.12$ in our baseline ensemble), the horizon-limited volume spectrum concentrates onto a small set of discrete outcomes and exhibits a pronounced forbidden gap separating small, localized excitations from rapidly spreading ``shockwave'' events.
Above this window ($p_{\mathrm f}\approx 0.14$), localized outcomes are strongly suppressed and propagation becomes predominantly explosive.
These results support a picture in which consistency-generated causal geometry behaves like a disordered medium with blocked regions, conductive channels, and rare trapping structures that produce short-lived quasi-particle resonances without imposing continuum field equations.
\end{abstract}

\maketitle

\paragraph{Model.}
We generate directed graphs by placing $N$ vertices into $L$ layers and sampling forward edges with probability $p_{\mathrm f}$ plus skip edges ($p_{\mathrm{skip}2}$, $p_{\mathrm{skip}3}$).
We then apply a simulated annealing ``genesis'' phase that rewires edges to reduce a local motif-based energy (the specific motif penalties are documented in the code release).
This procedure is intentionally minimal: there are no continuum assumptions, no action functional on a manifold, and no tuned dispersion relation.
After genesis we optionally inject a compact defect by selecting $k$ vertices and densifying their mutual connectivity (a ``knot'').

To quantify emergent geometry we measure the discrete light cone $V_s(t)$: the number of vertices reachable from a source $s$ within $t$ directed steps.
For polynomial growth $V_s(t)\propto t^{d}$ we estimate $d_{\mathrm{eff}}$ by fitting $\log V$ vs.\ $\log t$ on $t\in[1,8]$.
We also compute geodesic focusing metrics (``lensing'') by comparing shortest-path distances to a target in the presence/absence of injected defects.

\paragraph{Transport regimes and a discrete spectrum near criticality.}
A key diagnostic is the \emph{horizon-limited} light-cone volume $V_s(t^\star)$ at a fixed horizon $t^\star$ (default $t^\star=8$ in this work).
Across random sources, $V_s(t^\star)$ serves as an operational proxy for how a small perturbation can spread through the causal medium over a fixed proper time.
We define three regimes using fixed thresholds:
(i) \emph{dead} if $V_s(t^\star)<5$,
(ii) \emph{localized} if $5\le V_s(t^\star)\le 20$,
and (iii) \emph{shockwave} if $V_s(t^\star)>20$.

In a reproducible batch of $5$ seeds at each density (with $8$ sources per seed; $N=500$, $L=100$, $p_{\mathrm{skip}2}=0.02$, $p_{\mathrm{skip}3}=0.005$), we find:
at $p_{\mathrm f}=0.12$ the outcomes split into dead ($42.5\%$), localized ($17.5\%$), and shockwave ($40.0\%$) events, while at $p_{\mathrm f}=0.14$ localized outcomes drop to $5.0\%$ and the remaining events are evenly split between dead and shockwave.
More strikingly, the \emph{set of observed} $V_s(t^\star)$ values at $p_{\mathrm f}=0.12$ is highly discrete and exhibits a large forbidden interval:
in our sample, the largest consecutive gap is between $V=23$ and $V=77$ (no outcomes in $24\le V\le 76$).
At $p_{\mathrm f}=0.14$ a different bimodality appears, with a gap between $V=10$ and $V=99$.
The existence of such forbidden intervals is not expected from a naive ``random noise'' picture and motivates interpreting the critical regime as supporting a small number of transport resonances (quasi-particles) whose sizes are constrained by the causal geometry.

\paragraph{Localized activity pockets from defect injection.}
To complement purely topological transport probes, we evolve a binary field on the post-genesis graph using a simple deterministic threshold update rule.
When a compact defect is injected, the dynamics frequently exhibits persistent \emph{activity pockets} detected by connected-component persistence in spacetime slices.
These pockets are metastable: they persist across the simulated time window but need not survive indefinitely without a conserved topological charge.
This distinction is important for interpretation: in our framework, ``glider-like'' visual signals correspond to energy confined to naturally occurring conductive channels in the graph, rather than to a hand-engineered rule designed to force a particle to move.

\paragraph{Interpretation and outlook.}
The combined evidence supports a ``semi-conductor'' picture of the consistency-generated vacuum: most regions are effectively insulating (dead), some channels are conductive (shockwave), and rare structures trap perturbations into localized resonances.
This is qualitatively reminiscent of Anderson localization in disordered media~\cite{anderson1958}.
Determining the order of the transition and extracting localization lengths will require finite-size scaling and additional null-model comparisons, both of which are straightforward within the released suite.

Finally, independent symmetry experiments in this codebase compute finite braid-image groups in small-$N$ sectors; a representative $N=3$ configuration yields a non-Abelian group of order $26208$, matching $|\mathrm{GL}(2,\mathbb{F}_{13})|$.
We view this as a promising indicator that non-trivial internal symmetries can emerge from local consistency constraints, but we treat any mapping to continuum gauge structure as a hypothesis for future work.

\begin{acknowledgments}
Code and reproducibility materials are included with this submission.
\end{acknowledgments}

\bibliographystyle{apsrev4-2}
\bibliography{refs}

\end{document}
